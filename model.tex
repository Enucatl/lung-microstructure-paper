\section{Modelling the polychromatic dark-field signal}
Previous works [Lynch, Gkoumas] have demonstrated quantitative estimation of
the dark-field signal for a monochromatic source and monodisperse solutions
of micrometer-sized spheres. In particular, the coefficient $\mu_B$, for a
beam of wavelength $\lambda$ was estimated as:

\begin{equation}
\mu_B = \frac{3\pi}{\lambda^2}f |\Delta n|^2 d
    \begin{cases}
    D' & \text{if } D' \leqslant 1\\[2ex]
    \\!\begin{align}
    & D' - \sqrt{D'^2 - 1}\\
    & (1 + D'^{-2}/2) \\
    & + (D'^{-1} + D'^{-3} / 4) \\
    & \log\left(\frac{D' + \sqrt{D'^2 - 1}}{D' - \sqrt{D'^2 - 1}}\right)
    \end{align} & \text{otherwise.}
    \end{cases}
    \label{eqn:lynch}
\end{equation}

The autocorrelation length is $d = L\lambda / p$ with $L$ the sample to
detector distance, $p$ and the period of $G_2$. $D'$ is a normalized
particle diameter equal to $D/d$, where $D$ is the particle diameter. $f$ is
the fraction of volume occupied by the scattering material and $n = 1 -
\delta - i\beta$ is the complex refractive index.

We rewrite this formula more concisely as

\begin{equation}
    \mu_B(\energy) = C |\Delta n(\energy)|^2 \energy u(\energy),
    \label{eqn:lynchshort}
\end{equation}
where $\energy$ is the energy of the beam, $C = 3 fL / 4p$ is a constant depending only on the setup geometry and the volume fraction $f$ occupied by the spheres.


In the case of a polychromatic source, we tested a model where different
energies interact with the sample independently of each other, thus allowing
an incoherent sum of the dark-field signals over the spectral weights
$s(\energy)$:

\begin{equation}
    R = C \frac{\sum_\energy s(\energy)|\Delta n(\energy)|^2 \energy u(\energy)}{\sum_\energy s(\energy) \energy \beta}.
    \label{eqn:lynchpolychromatic}
\end{equation}

This formula was first tested on monodisperse solution of silicon dioxide
spheres with sizes ranging from 0.166 μm to 7.760 μm, with a volume fraction of 20\%
in glycerine. This model yields a good agreement between the continuous line
and the experimental data representing radiographies of the silicon dioxide
sphere samples recorded on the laboratory source interferometer.

[plot]

A model of the sample is required to calculate the spectral weights
$s(\energy)$ in order to avoid beam hardening artefacts. The source spectrum is
simulated with the SpekCalc~\cite{spekcalc} software, then attenuated
according to NIST absorption tables for the sample in the beam path and the
efficiency of the detector.

In this work, we propose to model the lung sample as a solution of
polydisperse spherical air bubbles embedded in tissue. The distribution of
the sphere diameters is extracted from the microtomographic data, together
with the fraction of the volume occupied by the tissue. The density of the
tissue is also estimated from tomographic projections: the composition is
taken from the ICRU-44 lung tissue tables, while the density is increased to
1.6 g/cm$^3$ as to match the absorption values recorded on the tomographic
projections. This results from the drying procedure for the fixation of
the samples, leaving a denser material than in \emph{in vivo} conditions.

The coefficient $\mu_B$ can then be expressed as a double summation over the spectrum
$s(\energy)$ and over the distribution of sphere diameters $\rho(d)$:

\begin{equation}
    \mu_B = \sum_{d}\sum_{\energy} \mu_B(r, E; n, f)\rho(d)s(\energy)
    \label{eqn:totalsum}.
\end{equation}
