\section{Results}\label{sec:model}
\subsection{Model
Previous works~\cite{Lynch:11,Gkoumas2016} have demonstrated quantitative estimation of
the dark-field signal for a monochromatic source and monodisperse solutions
of micrometer-sized spheres. In particular, the coefficient $\mu_B$, for a
beam of wavelength $\lambda$ is:

\begin{equation}
\mu_B = \frac{3\pi}{\lambda^2}f |\Delta n|^2 d
    \begin{cases}
    D' & \text{if } D' \leqslant 1\\
    \begin{align}
    & D' - \sqrt{D'^2 - 1}\\
    & (1 + D'^{-2}/2) \\
    & + (D'^{-1} + D'^{-3} / 4) \\
    & \log\left(\frac{D' + \sqrt{D'^2 - 1}}{D' - \sqrt{D'^2 - 1}}\right)
    \end{align} & \text{otherwise.}
    \end{cases}
    \label{eqn:lynch}
\end{equation}

The autocorrelation length is $d = L\lambda / p$ with $L$ the sample to
detector distance, $p$ and the period of $G_2$. $D'$ is a normalized
particle diameter equal to $D/d$, where $D$ is the particle diameter. $f$ is
the fraction of volume occupied by the scattering material and $n = 1 -
\delta - i\beta$ is the complex refractive index.

We rewrite this formula more concisely as

\begin{equation}
    \mu_B(\energy) = C |\Delta n(\energy)|^2 \energy u(\energy),
    \label{eqn:lynchshort}
\end{equation}
where $\energy$ is the energy of the beam, $C = 3 fL / 4p$ is a constant
depending only on the setup geometry and the volume fraction $f$ occupied by
the spheres and $\energy$ is the energy of the beam.


In the case of a polychromatic source, we tested a model where different
energies interact with the sample independently of each other, thus allowing
an incoherent sum of the dark-field signals over the spectral weights
$s(\energy)$:

\begin{equation}
    R = C \frac{\sum_\energy s(\energy)|\Delta n(\energy)|^2 \energy u(\energy)}{\sum_\energy s(\energy) \energy \beta}.
    \label{eqn:lynchpolychromatic}
\end{equation}

This formula was first tested on monodisperse solution of silicon dioxide
spheres with sizes ranging from 0.166 μm to 7.760 μm, with a volume fraction of 20\%
in glycerine. This model yields a good agreement between the continuous line
and the experimental data representing radiographies of the silicon dioxide
sphere samples recorded on the laboratory source interferometer.