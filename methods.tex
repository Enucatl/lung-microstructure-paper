\section{Methods}
\subsection{Sample preparation}
Eight \emph{ex-vivo} mouse samples were prepared at the university of Bern.
In order to preserve the delicate structure of the lungs, critical point
drying is used in order to avoid the damage caused by evaporating the liquids
from the sample in ordinary pressure conditions.

[ETHICAL DETAILS? PREPARATION STEPS?]

\subsection{Image acquisition}
The microtomography high-resolution 3D images were acquired at the X02DA
TOMCAT beamline of the Swiss Light Source at the Paul Scherrer Institute
(Villigen, Switzerland). The X-ray beam is generated with a 2.9 T bending
magnet from electrons at an energy of 2.4 GeV. The current in the storage
ring is 400 mA, top-up mode. Two multilayer monochromator crystals are used
to filter X-rays with an energy of 11 keV. A 20 μm thick scintillator
converts the X-rays into visible light, collected by a 10x objective onto a
high-speed CMOS sensor with an effective pixel size of $0.65 \times 0.65$
μm$^2$. The exposure time for each tomographic scan was set at 100 ms per
projection, with 1801 projections.

The radiographies on the laboratory source were taken on a symmetric
Talbot-Lau interferometer with a grating pitch of 5.4 μm and an intergrating
distance of 26 cm, the phase-shift grating provides a phase shift of $\pi$
at 45 keV. The average visibility of the interference pattern without any
sample is 8\%. The source is a Comet MXR-225/26 X-ray tube operated at 100
kV and 6 mA. The source size is 1 mm. The detector is a prototype based on
Santid CdTe by Dectris Ltd. The CdTe 750 μm sensor provides high quantum
efficiency at high energies (>90\% at 60 keV) with a pixel size of 75 x 75
μm$^2$. The phase-stepping procedure was performed with 31 phase steps with
1 s exposure per step.

\subsection{Processing of the tomographic data}
The tomographic sinograms are preprocessed with the Paganin algorithm and
the 3D volume is then reconstructed with the gridrec algorithm. The volumes
are then thresholded with the Otsu algorithm, with independent thresholds
for each slice, to provide a binary labelling for tissue and air. A cycle of
erosion and dilation is applied to remove single pixel artifacts while still
keeping the septa between neighboring alveoli. This preserves the structures
in the lung because septa, while possibly being only one pixel in thickness,
are also connected to surrounding tissue. On the other hand, isolated pixels
should be removed as they could affect the subsequent analysis.

Next, each lung sample is manually stitched from three or four local
tomographies, depending on the sample thickness, and the local air thickness
map is calculated. The local thickness map is an algorithm developed for
bone analysis or foam analysis in material science, that aims at calculating
the maximum diameter of a sphere fitting in each hole in the sample.

The euclidean distance transform is calculated from the image, yielding the
distance of each point in the airways from the closest wall. The points in
the transformed dataset are then sorted in descending order in order to fit
the largest possible sphere at each location.

In the case of lungs, each of the airways is filled with the largest possible
sphere, resulting in a model where the lung airways are a collection of
spheres embedded in the background of the tissue.

To obtain a distribution of sphere sizes, kernel density estimation in the R
statistics package by Duong is used in order to avoid possible artifacts
resulting from the arbitrary binning of the histogram of sphere sizes.

\subsection{Processing of the radiographic data}
Radiographic data in the Talbot-Lau interferometer are recorded with the
following procedure. The interferometer is composed of three gratings,
labelled $G_0$, $G_1$ and $G_2$. The first grating $G_0$ creates an array of
individually coherent but mutually incoherent light sources from the
focal spot of the X-ray tube. The second grating $G_1$ generates an
interference pattern downstream by exploiting the self-imaging Talbot
effect. This interference pattern has the same periodicity of the grating
itself, 5.4 μm for the current experiment, and it is therefore too small to
be resolved by the detector. A third grating $G_2$ with the same period is
then moved in a direction perpendicular to the interference fringes over one
period in order to sample the intensity of the pattern at several positions.
This results in a so-called phase-stepping curve whose parameters change
when a sample is present in the beam: the average intensity is reduced
according to the absorption of the X-rays in the sample, while the reduction
in the amplitude of the curve is determined by microstructures.

A sinusoidal fit is performed with Fourier component analysis, where the
reference curve for each pixel is:

\begin{equation}
    c_f(x) = a_{0,f} + a_{1,f} \sin(x + \phi_{f}),
    \label{eqn:flat}
\end{equation}

the parameters are changed when a sample is in the beam:

\begin{equation}
    c_s(x) = a_{0,s} + a_{1,s} \sin{x + \phi_{s}).
    \label{eqn:sample}
\end{equation}

For each pixel, the transmission image $A = a_{0,s} / a_{0,f}$ and the
dark-field image $B = a_{1,s} / Aa_{1,f}$ are calculated. Both of these
quantities are known to follow the Beer-Lambert law, with an exponential
decay related to the thickness $t$ of the sample: $A_\exp(-\mu_A t)$, $B =
\exp(-\mu_B t)$. Since the lung samples are irregular in shape, it is
beneficial to examine the ratio of the logarithms $R = \log(B) / \log(A)$ in
order to remove the local dependence on the thickness of the sample in the
radiography.
