\section{Methods}\label{sec:methods}
\subsection{Sample preparation}
Eight \emph{ex-vivo} mouse samples were prepared at the university of Bern.
In order to preserve the delicate structure of the lungs, critical point
drying is used in order to avoid the damage caused by evaporating the liquids
from the sample in ordinary pressure conditions.

[ethical details? preparation steps?]

\subsection{Image acquisition}\label{sec:acquisition}
The high-resolution microtomographic 3D image data were acquired at the X02DA
TOMCAT beamline of the Swiss Light Source at the Paul Scherrer Institute
(Villigen, Switzerland). The X-ray beam was generated with a 2.9 T superbending
magnet from electrons at an energy of 2.4 GeV with 
ring is 400 mA, top-up mode. Two multilayer monochromator crystals are used
to filter X-rays with an energy of 11 keV. A 20 μm thick scintillator
converts the X-rays into visible light, collected by a 10x objective onto a
high-speed CMOS sensor with an effective pixel size of $0.65 \times 0.65$
μm$^2$. The exposure time for each tomographic scan was set at 100 ms per
projection, with 1801 projections.

The radiographies on the laboratory source were taken on a symmetric
Talbot-Lau interferometer (see fig.~\ref{248327}) with a grating pitch of 5.4 μm and an intergrating
distance of 26 cm, the phase-shift grating provides a phase shift of $\pi$
at 45 keV. The average visibility of the interference pattern without any
sample is 14\%. The source is a Comet MXR-225/26 X-ray tube operated at 100
kV and 6 mA. The source size is 1 mm. The detector is a prototype based on
Santid CdTe by Dectris Ltd. The CdTe 750 μm sensor provides high quantum
efficiency at high energies (>90\% at 60 keV) with a pixel size of 75 x 75
μm$^2$. The phase-stepping procedure was performed with 31 phase steps with
1 s exposure per step.
