\section{Methods}\label{sec:methods}
\subsection{Sample preparation}
Wild-type 129S6 and SvEv/Tac mice (Dai et al., 2015) were exposed to cigarette smoke for 5 hours per day and 5 days per week for 4 months. This treatment induced a very mild pulmonary emphysema. Controls were exposed to filtered air. Smoke of 3R4F research cigarettes (University of Kentucky, Lexington, KY) was generated as previously described (Cremona et al., 2013) by a TE10z smoking machine (Teague Enterprises, Woodland, CA) connected to whole-body exposure chambers. The animals were housed in the central animal facility of the University of Bern at a 12/12 hour day/night circle. They received water and food ad libitum. After 4 months of smoking lungs were instilled with 1.5\% paraformaldehyde-1.5% glutaraldehyde in 0.15 M HEPES pH 7.35 through a tracheal cannula at a constant pressure of 20 cm H2O. After ligating the trachea the lungs were removed and placed in fresh fixative for at least 24 hours to complete fixation (Cremona et al., 2013). In order to preserve the delicate structure of the lungs and to prepare them for X-ray imaging, the lungs were washed in PBS and a graduated series of ethanol, followed by critical point drying (Barre et al., 2016; Kaeslin et al., 2005).

Handling of the animals before and during the experiments, as well as the experiments themselves, were approved and supervised by the Swiss Agency for Environment, Forests and Landscape and the Veterinary Service of the Canton of Berne. For ethical reasons we were obliged to keep the number of animals as low as possible.

\subsection{Image acquisition}\label{sec:acquisition}
The high-resolution microtomographic 3D image data were acquired at the X02DA TOMCAT beamline of the Swiss Light Source at the Paul Scherrer Institute (Villigen, Switzerland). The X-ray beam was generated with a 2.9 T superbending magnet from electrons at an energy of 2.4 GeV with synchrotron ring current of 400 mA in top-up mode. A monochromatic X-ray energy of 11 keV was tuned with a double-multilayer monochromator. After penetrating the sample, a 20 μm thick scintillator (LuAg:Ce) was used to convert the X-rays into visible light and combined with a 10x magifying visible-light objective to yield an effective pixel size of $0.65 \times 0.65\,$μm$^2$. A scientific CMOS detector (pco.Edge 5.5) was used with a single-projection exposure time of 100 ms and a total of 1801 tomographic projections. A sample-to-detector distance of approximately 10 mm was employed to yield a good compromise between contrast and spatial resolution and, prior to CT-reconstruction \cite{Marone2012}, phase retrieval was conducted from single defocused images using a transport-of-intensity approximation as originally proposed by Paganin \textit{et al.} \cite{Paganin2002}. 

The radiographies on the laboratory source were taken on a symmetric
Talbot-Lau interferometer (see fig.~\ref{248327}) with a grating pitch of 5.4 μm and an intergrating
distance of 26 cm, the phase-shift grating provides a phase shift of $\pi$
at 45 keV. The average visibility of the interference pattern without any
sample is 14\%. The source is a Comet MXR-225/26 X-ray tube operated at 100
kV and 6 mA. The source size is 1 mm. The detector is a prototype based on
Santid CdTe by Dectris Ltd. The CdTe 750 μm sensor provides high quantum
efficiency at high energies (>90\% at 60 keV) with a pixel size of 75 x 75
μm$^2$. The phase-stepping procedure was performed with 31 phase steps with
1 s exposure per step.
