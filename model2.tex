
A model of the sample is required to calculate the spectral weights
$s(\energy)$ in order to avoid beam hardening artefacts. The source spectrum is
simulated with the SpekCalc~\cite{spekcalc} software, then attenuated
according to NIST absorption tables for the sample in the beam path and the
efficiency of the detector.

In this work, we propose to model the lung sample as a solution of
polydisperse spherical air bubbles embedded in tissue. The distribution of
the sphere diameters is extracted from the microtomographic data, together
with the fraction of the volume occupied by the tissue. The density of the
tissue is also estimated from tomographic projections: the composition is
taken from the ICRU-44 lung tissue tables~\cite{White_1989}, while the density is increased to
1.6 g/cm$^3$ as to match the absorption values recorded on the tomographic
projections. This results from the drying procedure for the fixation of
the samples, leaving a denser material than in \emph{in vivo} conditions.

The coefficient $\mu_B$ can then be expressed as a double summation over the spectrum
$s(\energy)$ and over the distribution of sphere diameters $\rho(d)$:

\begin{equation}
    \mu_{B,\text{total}} = \sum_{d}\sum_{\energy} \mu_B(r, E; n, f)\rho(d)s(\energy)
    \label{eqn:totalsum}.
\end{equation}
