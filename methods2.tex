
\subsection{Processing of the tomographic data}\label{sec:tomoprocessing}
The tomographic sinograms are preprocessed with the Paganin
algorithm~\cite{Paganin_2002a} and
the 3D volume is then reconstructed with the gridrec
algorithm~\cite{Marone_2012}. The volumes, shown in
figure~\ref{fig:reconstructed} are then thresholded with the Otsu
algorithm~\cite{Otsu_1979}, with independent thresholds for each slice, to
provide a binary labelling for tissue and air. A cycle of
erosion and dilation is applied to remove single pixel artifacts while still
keeping the septa between neighboring alveoli (figure~\ref{591939}).
This preserves the structures in the lung because septa, while possibly
being only one pixel in thickness, are also connected to surrounding tissue.
On the other hand, isolated pixels should be removed as they could affect
the subsequent analysis.