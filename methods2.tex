
\subsection{Processing of the tomographic data}\label{sec:tomoprocessing}
The tomographic sinograms are preprocessed with the Paganin
algorithm~\cite{Paganin_2002} and
the 3D volume is then reconstructed with the gridrec
algorithm~\cite{Marone:pp5022}. The volumes
are then thresholded with the Otsu algorithm~\cite{Otsu_1979}, with independent thresholds
for each slice, to provide a binary labelling for tissue and air. A cycle of
erosion and dilation is applied to remove single pixel artifacts while still
keeping the septa between neighboring alveoli. This preserves the structures
in the lung because septa, while possibly being only one pixel in thickness,
are also connected to surrounding tissue. On the other hand, isolated pixels
should be removed as they could affect the subsequent analysis.

Next, each lung sample is manually stitched from three or four local
tomographies, depending on the sample thickness, and the local air thickness
map is calculated. The local thickness map is an algorithm developed for
bone analysis or foam analysis in material science~\cite{6778077}, that aims
at calculating the maximum diameter of a sphere fitting in each hole in the
sample.

The euclidean distance transform is calculated from the image, yielding the
distance of each point in the airways from the closest wall. The points in
the transformed dataset are then sorted in descending order in order to fit
the largest possible sphere at each location.

In the case of lungs, each of the airways is filled with the largest possible
sphere, resulting in a model where the lung airways are a collection of
spheres embedded in the background of the tissue.

To obtain a distribution of sphere sizes, kernel density estimation in the R
statistics package \emph{ks}~\cite{JSSv021i07} is used in order to avoid possible artifacts
resulting from the arbitrary binning of the histogram of sphere sizes.

\subsection{Processing of the radiographic data}\label{sec:radioprocessing}
Radiographic data in the Talbot-Lau interferometer are recorded with the
following procedure. The interferometer is composed of three gratings,
labelled $G_0$, $G_1$ and $G_2$. The first grating $G_0$ creates an array of
individually coherent but mutually incoherent light sources from the
focal spot of the X-ray tube. The second grating $G_1$ generates an
interference pattern downstream by exploiting the self-imaging Talbot
effect. This interference pattern has the same periodicity of the grating
itself, 5.4 μm for the current experiment, and it is therefore too small to
be resolved by the detector. A third grating $G_2$ with the same period is
then moved in a direction perpendicular to the interference fringes over one
period in order to sample the intensity of the pattern at several positions.
This results in a so-called phase-stepping curve whose parameters change
when a sample is present in the beam: the average intensity is reduced
according to the absorption of the X-rays in the sample, while the reduction
in the amplitude of the curve is determined by microstructures.

A sinusoidal fit is performed with Fourier component analysis, where the
reference curve for each pixel is:

\begin{equation}
    c_f(x) = a_{0,f} + a_{1,f} \sin(x + \theta_{f}),
    \label{eqn:flat}
\end{equation}

the parameters are changed when a sample is in the beam:

\begin{equation}
    c_s(x) = a_{0,s} + a_{1,s} \sin(x + \theta_{s}).
    \label{eqn:sample}
\end{equation}