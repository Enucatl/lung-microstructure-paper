
Next, each lung sample is manually stitched from three or four local
tomographies, depending on the sample thickness, and the local air thickness
map is calculated. The local thickness map is an algorithm developed for
bone analysis or foam analysis in material science~\cite{Hildebrand_2003}, that aims
at calculating the maximum diameter of a sphere fitting in each hole in the
sample.
The euclidean distance transform is calculated from the image, yielding the
distance of each point in the airways from the closest
wall.
The points in the transformed dataset are then sorted in descending order in
order to fit the largest possible sphere at each location.
Each of the airways is filled with the largest possible
sphere, resulting in a model where the lung airways are a collection of
spheres embedded in the background of the tissue.


