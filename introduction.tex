\section{Introduction}\label{sec:introduction}
X-ray grating interferometry has been developed over the course of the last
fifteen years on both synchrotron and laboratory
sources~\cite{David_2002,1347-4065-42-7B-L866,Weitkamp_2005,1347-4065-45-6R-5254,Pfeiffer2006}. Interferometric
imaging allows simultaneous eaccess to three independent images: the
conventional absorption image, a differential phase signal and a dark-field
signal, also known as visibility contrast. This last signal has been
reported by multiple sources as being quantitatively linked to the presence
of unresolved structures in the sample, much smaller than the pixel size of
the detector, and typically of the order of the
micrometer~\cite{Pfeiffer2008,Lynch:11,Yashiro:10}.
Various clinically relevant
applications~\cite{Wen_2009,Thilo2013} have been
proposed, including lung microstructural analyses~\cite{Meinel2013,Schleede17880}.

In particular, emphysema is a pathological condition of the lung
resulting in structural changes in the alveoli, that is at the smallest
hierarchical level in the tissue. These changes are caused by the
destruction of interalveolar septa, with larger air spaces progressively
replacing the fine architecture of the lung parenchyma~\cite{Sharafkhaneh_2008}. These larger spaces
have a less favorable surface-to-volume ratio, thus lowering the efficiency
of respiratory exchanges. Absorption radiography proved to be accurate in
diagnosing emphysema only in the advanced stages of the disease. 
High-resolution computed tomography is able to detect regions in the lung
with abnormally low attenuation, at the cost of exposing the patient
to a higher radiation dose.

Previous studies on murine lungs established that the increased sensitivity of
the dark-field signal of an X-ray grating interferometer to micrometer-sized
features can distinguish between emphysematous from healthy samples and
provide a mapping of the parenchyma showing the localization of the
structural damage.
The strength of the dark-field signal in grating interferometry is
given by the small-angle scattering of X-rays by structures smaller than the
spatial resolution of the imaging system. The lung is therefore an ideal
application for this technique, since the alveoli are much
smaller than the spatial resolution available for chest radiography.

In this work we present a quantitative model of the dark-field signal
generated by lung tissue in an X-ray grating interferometer on a laboratory
source. High-resolution tomographic data resolving the features down to single
alveoli is analyzed with established post-processing techniques to extract a
ground truth on the sizes of the structures composing the lung tissue. The
lung is then modeled as a suspension of spheres of different sizes in a
homogeneous medium, in the hypothesis that the signals generated by spheres
of different sizes and by X-rays of different energies sum up incoherently.
Finally, an X-ray interferometric radiography on a laboratory source is
recorded, to allow a direct comparison between the expected dark-field
signal calculated according to this model and the experimental values.
