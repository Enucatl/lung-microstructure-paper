
To obtain a distribution of sphere sizes, kernel density estimation in the R
statistics package \emph{ks}~\cite{Duong_2007} is used in order to avoid possible artifacts
resulting from the arbitrary binning of the histogram of sphere sizes. 


\subsection{Processing of the radiographic data}\label{sec:radioprocessing}
Radiographic data in the Talbot-Lau interferometer are recorded with the
following procedure. The interferometer is composed of three gratings,
labelled $G_0$, $G_1$ and $G_2$. The first grating $G_0$ creates an array of
individually coherent but mutually incoherent light sources from the
focal spot of the X-ray tube. The second grating $G_1$ generates an
interference pattern downstream by exploiting the self-imaging Talbot
effect. This interference pattern has the same periodicity of the grating
itself, 5.4 μm for the current experiment, and it is therefore too small to
be resolved by the detector. A third grating $G_2$ with the same period is
then moved in a direction perpendicular to the interference fringes over one
period in order to sample the intensity of the pattern at several positions.
This results in a so-called phase-stepping curve whose parameters change
when a sample is present in the beam: the average intensity is reduced
according to the absorption of the X-rays in the sample, while the reduction
in the amplitude of the curve is determined by microstructures.

A sinusoidal fit is performed with Fourier component analysis, where the
reference curve for each pixel is:

\begin{equation}
    c_f(x) = a_{0,f} + a_{1,f} \sin(x + \theta_{f}),
    \label{eqn:flat}
\end{equation}

the parameters are changed when a sample is in the beam:

\begin{equation}
    c_s(x) = a_{0,s} + a_{1,s} \sin(x + \theta_{s}).
    \label{eqn:sample}
\end{equation}