
The measured data agreed with the
theory for all samples, and the $\chi^2$ statistic can be calculated

\begin{equation}
    \chi^2 = \sum_{i=1}^8 \dfrac{(R_{\text{obs}} -
    R_{\text{th}})Ъ}{\mathop{\mathrm{Var}}(R_{\text{obs}})} = 2.31.
    \label{eqn:chisq}
\end{equation}

With 7 degrees of freedom, this results in a right-tail probability of 0.94 that the 
observations are consistent with the model, thus validating the quantitative estimation of
dark-field values for a lung model as a collection of incoherently
superimposed microscopic spheres.

Possible sources of inaccuracy of our model include a significantly
different distribution of shapes in the lung parenchyma, or additional
effects of beam hardening which are known to influence the dark-field
signal. In order to consider the effects of possible shape asymmetries, a
mouse lung fixed with the same procedure described in section~\ref{sec:methods} was scanned at the
X02DA TOMCAT beamline with the setup described in [Matias omnidirectional].
This technique is similar to Talbot interferometry, but it is able to detect
scattering from unresolved microscopic structures under different angles in
a single shot by
creating an array of circular interference patterns. The intensity of the
dark-field signals for different angles can then be recovered for each
pixel, and an asymmetry value defined as
\begin{equation}
    \alpha =
    \frac{\max_\theta{B(\theta)} -
    \min_\theta{B(\theta)}}{\max_\theta{B(\theta)} + \min_\theta{B(\theta)}}
    \label{eqn:asymmetry}
\end{equation}
can be calculated for each pixel. This allows us to exclude the possibility
that there are significant inhomogeneities in the dark-field response of the
lung structures under different angles, which would directly invalidate the
model of a superposition of spheres. The results are shown in figure [] and
as a histogram [], demonstrating a limited asymmetry over the whole sample.

Another effect commonly reported in dark-field analyses on wide, polychromatic
sources is the influence of beam hardening on the recorded dark-field
values. In our case, given the high voltage of the source and the small
thickness of the samples (less than 4 mm), the absorption ranges from 4\% to
6\%. The correction to dark-field values provided by [yashiro] is therefore
not applied, as it is reported to be relevant for samples absorbing at least
50\% of the incoming light. The beam hardening effect is considered insofar
it affects the spectral weights of eq.~\ref{eqn:totalsum}, which are calculated
after the sample.
