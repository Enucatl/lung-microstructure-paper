For each pixel, the transmission image $A$ and the
dark-field image $B$ are calculated (*). Both of these
quantities are known to follow the Beer-Lambert law, with an exponential
decay related to the thickness $t$ of the sample: $A = \exp(-\mu_A t)$, $B =
\exp(-\mu_B t)$. Since the lung samples are irregular in shape, it is
beneficial to examine the ratio of the logarithms $R$ in
order to remove the local dependence on the thickness of the sample in the
radiography:

\begin{align*}
    A &= \dfrac{a_{0,s}}{a_{0,f}}\\
    B &= \dfrac{a_{1,s}}{a_{1,f}}\dfrac{a_{0,f}}{a_{0,s}}\\
    R &= \dfrac{\log(B)}{\log(A)}.
    \label{eqn:definitions}
\end{align*}
