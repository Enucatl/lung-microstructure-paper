The section of the radiographic image of each lung sample is manually
matched to the local tomography, as identified during the alignment of the
tomographic scan. The average and standard deviation of the $R$ values are
calculated and plotted in fig. [] (black dots and errorbars). The expected
values according to formula [] are calculated with the inputs from the
microtomographic datasets, as described in section [], and are plotted on
fig. [] with red dots for each sample. The measured data agreed with the
theory for all samples, thus validating the quantitative estimation of
dark-field values for a lung model as a collection of incoherently
superimposed microscopic spheres.

Additional sample scans: a mouse lung fixed with the same procedure
described in [] was scanned at the X02DA TOMCAT beamline with the setup
described in [Matias omnidirectional]. This technique is similar to Talbot
interferometry, but it is able to detect scattering from unresolved
microscopic structures at different angles by creating an array of circular
interference patterns. The intensity of the dark-field signals under
different angles can then be recovered for each pixel, and an asymmetry
score defined as $\alpha = \frac{\max_\theta{B(\theta)} -
\min_\theta{B(\theta)}}{\max_\theta{B(\theta)} + \min_\theta{B(\theta)}}$ can
be calculated for each pixel. This allows us to exclude the possibility that
there are significant inhomogeneities in the dark-field response of the lung
structures under different angles, which would directly invalidate the model
of a superposition of spheres. The results are shown in figure [] and as a
histogram [], demonstrating a limited asymmetry over the whole sample.
